\chapter{Аналитическая часть}

\section{Постановка задачи}

В соответствии с заданием на курсовую работу необходимо разработать модуль ядра 
для подмены пользовательских данных, передаваемых на USB-накопители. 
Подмена данных осуществляется изменением значения каждого байта данных,
с использованием XOR операции. Для достижения поставленной цели необходимо решить
следующие задачи:

\begin{enumerate}
    \item провести обзор функций и структур ядра, представляющих возможность 
    реализации разрабатываемого модуля;
    \item провести анализ методов перехвата управления у функций ядра; 
    \item провести анализ файловых систем внешних накопителей;
    \item реализовать и протестировать модуль.
\end{enumerate}

\section{Архитектура подсистемы USB в Linux}

Основные компоненты, реализующие работу с USB-устройствами~\cite{MasterUSB}:

\begin{enumerate}
    \item драйвер устройства;
    \item ядро USB (USB Core) -- общий код, управляющий всей подсистемой;
    \item драйвер хост-контроллера (HCD) -- реализует программно-аппаратное взаимодействие.
\end{enumerate}

Схема взаимодействия драйверов USB:

\begin{enumerate}
\item драйвер устройства создает или получает запрос на передачу информации;
\item драйвер инициализирует запрос всей необходимой информацией:
    \begin{itemize}[label=---]
        \item тип запроса (чтение, запись, управление);
        \item адрес устройства и номер конечной точки (endpoint);
        \item указатель на буфер данных;
        \item размер буфера данных;
        \item функция обратного вызова (callback) -- которая будет вызвана, когда операция завершится.
    \end{itemize}
\item драйвер отправляет запрос в ядро USB;
\item ядро и HCD выполняют необходимую низкоуровневую работу, чтобы выполнить этот запрос на физической шине USB;
\item когда операция завершена (данные получены, отправлены или произошла ошибка), HCD вызывает функцию обратного 
вызова для этого запроса, чтобы уведомить драйвер устройства о результате.
\end{enumerate}

\section{Анализ основных структур ядра для работы с USB}

Для управления внешним накопителем, таким как USB-устройство в операционной 
системе представлено несколько структур.

\subsection{struct User Request Block (URB)}

URB --- структура, инкапсулирующая запрос на передачу данных, передаваемый 
драйвером USB-устройства низкоуровневому драйверу хост-контроллера.
Каждая операция ввода-вывода через USB шину оформляется в виде отдельного 
URB, что обеспечивает единый интерфейс для работы с различными типами 
USB-передач.

\subsubsection{Типы передач, поддерживаемые URB}

URB поддерживает все четыре типа передач, определенные в спецификации USB~\cite{ShinaUSB}:

\begin{enumerate}
\item управляющие посылки (Control Transfers), используемые для конфигурирования во время подключения 
и в процессе работы для управления устройствами.
\item сплошные передачи (Bulk Data Transfers) сравнительно больших пакетов без жестких требований 
ко времени доставки. Пакеты имеют поле данных размером 8, 16, 32 или 64 байт. Приоритет этих передач 
самый низкий, они могут приостанавливаться при большой загрузке шины.
\item прерывания (Interrupt) - короткие передачи, имеют спонтанный характер и должны обслуживаться не 
медленнее, чем того требует устройство. 
\item изохронные передачи (Isochronous Transfers) - непрерывные передачи в реальном времени, 
занимающие предварительно согласованную часть пропускной способности шины и имеющие заданную 
задержку доставки.
\end{enumerate}

\subsubsection{Ключевые поля структуры URB}

\begin{itemize}[label=---]
    \item struct usb\_device *dev -- представление устройства USB;
    \item int status -- статус выполнения;
    \item void *transfer\_buffer -- указатель на область данных для передачи;
    \item u32 transfer\_buffer\_length -- размер передаваемого буфера;
    \item struct scatterlist *sg -- указатель на начало списка разрозненных буферов для передачи;
    \item int num\_sgs -- длина списка разрозненных буферов;
    \item u32 actual\_length -- фактически переданное количество байт;
    \item usb\_complete\_t complete -- функция обратного вызова.
% разоббраться со флагами!!!!
\end{itemize}

\subsection{struct usb\_device}


%На~рисунке~\ref{img:test} символ семейства Unix-подобных операционных систем Linux.
%Он отличается от~<<обычных>> пингвинов желтым цветом клюва и~лап.

%\includeimage
%    {test} % Имя файла без расширения (файл должен быть расположен в директории inc/img/)
%    {f} % Обтекание (без обтекания)
%    {H} % Положение рисунка (см. figure из пакета float)
%    {0.25\textwidth} % Ширина рисунка
%    {Символ Linux (Tux)} % Подпись рисунка
%
%На~листингах \ref{lst:main.c} представлен исходный код программы Hello World на~языке программирования C в~двух вариантах оформления.
%
%\includelisting
%    {main.c} % Имя файла с расширением (файл должен быть расположен в директории inc/lst/)
%    {Исходный код программы Hello World} % Подпись листинга
%
%\includelistingpretty
%    {main.c} % Имя файла с расширением (файл должен быть расположен в директории inc/lst/)
%    {c} % Язык программирования (необязательный аргумент)
%    {Исходный код программы Hello World} % Подпись листинга