\chapter*{ВВЕДЕНИЕ}
\addcontentsline{toc}{chapter}{ВВЕДЕНИЕ}

Мониторинг активности сотрудников за рабочими станциями широко распространён, так как он позволяет как 
выявить неэффективное использование рабочего времени, так и уменьшить риски утечки служебной информации.

\textbf{Цель данной работы} --- провести обзор методов мониторинга активности пользователей на рабочих
станциях~\cite{NonAgentMonitoring}.

Для достижения данной цели необходимо решить следующие задачи:

\begin{itemize}[label=---]
	\item проанализировать предметную область мониторинга активности;
	\item выделить ключевые объекты наблюдения;
	\item описать методы решения, сформулировать критерии сравнения;
	\item сравнить перечисленные методы по сформулированным критериям.
\end{itemize}