\chapter{Технологический раздел}

\section{Выбор языка и среды программирования}

Для разработки модуля был выбран язык программирования C, так как на нём написано
ядро Linux. Для сборки модуля была использована утилита Make~\cite{Make}. 
В качестве среды разработки была использована среда VSCode~\cite{VSCode}.

\section{Конфигурация USB-накопителя}

На листинге~\ref{lst:usbconfig.c} представлены параметры отслеживаемого 
флеш накопителя, такие как идентификаторы производителя и устройства и 
имя тома, заданные в программе через макросы.

\includelisting
    {usbconfig.c}
    {Параметры отслеживаемого USB-накопителя}

\section{Функция проверки буфера на соответствие таблице FAT}

На листинге~\ref{lst:isfattable.c} представлен код функции,
проверяющей, что буфер является таблицей размещения файлов.

\includelisting
    {isfattable.c}
    {Функция проверки буфера на соответствие таблице FAT}

\section{Функция проверки буфера на заполненность нулями}

На листинге~\ref{lst:iszero.c} представлен код функции,
проверяющей, что буфер URB состоит из нулей.

\includelisting
    {iszero.c}
    {Функция проверки буфера на заполненность нулями}

\section{Функции, вызываемые при загрузке и выгрузке модуля}

На листинге~\ref{lst:modentries.c} представлен код функций,
выполняющихся при загрузке и выгрузке модуля. При входе регистрируется
\texttt{kprobe}, при выгрузке удаляется.

\includelisting
    {modentries.c}
    {Функции, вызываемые при загрузке и выгрузке модуля}

\section{Функция обработки структуры URB}

На листинге~\ref{lst:processurb.c} представлен код функции,
которая проверяет передаваемые в URB данные и меняет их с помощью XOR в случае,
если они являются внутренним содержимым передаваемых на USB-накопитель
файлов.

\includelisting
    {processurb.c}
    {Функция обработки структуры URB}

\section{Функция-обработчик точки останова}

На листинге~\ref{lst:kprobehandler.c} представлен код функции,
которая выполняется при достижении точки останова. Указатель на 
структуру URB берётся из регистра si. Проверяются
параметры устройства, направление передачи, является ли 
передаваемый URB блоком команды. После всех проверок 
вызывается функция обработки URB.

\includelisting
    {kprobehandler.c}
    {Функция-обработчик точки останова}

\section{Makefile}

Для сборки проекта была использована утилита Make и написан Makefile,
код которого представлен на листинге~\ref{lst:Makefile}.

\newpage

\includelisting
    {Makefile}
    {Код Makefile}

Выполнение команды make без параметров приведет к сборке файла 
загружаемого модуля ядра mon.ko. Для загрузки модуля 
используется команда sudo insmod mon.ko. Для выгрузки модуля 
используется команда sudo rmmod mon.