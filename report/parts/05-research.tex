\chapter{Исследовательский раздел}

\section{Исследование работы программы}

Для проверки работы модуля на USB-накопитель записывали файлы и папки с
разным содержимым.

\subsection{Тест №1 --- текстовый файл}

На флеш-накопитель был скопирован файл с содержимым, представленным
на листинге~\ref{lst:docker-compose.yml}:

\includelisting
    {docker-compose.yml}
    {Текст передаваемого файла}

На рисунке~\ref{img:TEST1-SIMPLE-FILE} видно, что содержимое файла
превратилось в набор произвольных байтов:

\includeimage
    {TEST1-SIMPLE-FILE}
    {f}
    {H}
    {1\textwidth}
    {Содержимое записанного на USB-накопитель файла}

\subsection{Тест №2 --- директория с поддиректориями и файлами}

На накопитель была скопирована директория с поддиректориями.
В поддиректориях хранились текстовые файлы.

При записи на накопитель структура папок осталась неизменной.
На листинге~\ref{lst:TEST2-DIRS.txt} представлена структура папок,
полученная с помощью команды \texttt{tree}.

\includelisting
    {TEST2-DIRS.txt}
    {Структура переданной папки}

Во всех текстовых файлах данные были изменены, что видно на 
рисунке~\ref{img:TEST2-DIRS}.

\includeimage
    {TEST2-DIRS}
    {f}
    {H}
    {1\textwidth}
    {Содержимое записанного на USB-накопитель файла}

\subsection{Тест №3 --- файл PDF}

На накопитель был скопирован одностраничный PDF файл.
При попытке открыть его с USB-накопителя с помощью стандартного приложения
Linux Ubuntu Просмоторщик Документов, было выведено следующее сообщение:

\includeimage
    {TEST3-binary}
    {f}
    {H}
    {1\textwidth}
    {Сообщение об ошибке при попытке открыть PDF-файл}

В результате записи на флешку была разрушена внутренняя структура файла.